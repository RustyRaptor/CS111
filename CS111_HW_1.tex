\documentclass{article}%
\usepackage{amsmath}%
\usepackage{amsfonts}%
\usepackage{amssymb}%
\usepackage{listings}
\usepackage{graphicx}
\usepackage{enumitem}
\usepackage{booktabs}
\usepackage{array}
%-------------------------------------------

\newlist{todolist}{itemize}{2}
\setlist[todolist]{label=$\checkmark$}
\newtheorem{theorem}{Theorem}
\newtheorem{acknowledgement}[theorem]{Acknowledgement}
\newtheorem{algorithm}[theorem]{Algorithm}
\newtheorem{axiom}[theorem]{Axiom}
\newtheorem{case}[theorem]{Case}
\newtheorem{claim}[theorem]{Claim}
\newtheorem{conclusion}[theorem]{Conclusion}
\newtheorem{condition}[theorem]{Condition}
\newtheorem{conjecture}[theorem]{Conjecture}
\newtheorem{corollary}[theorem]{Corollary}
\newtheorem{criterion}[theorem]{Criterion}
\newtheorem{definition}[theorem]{Definition}
\newtheorem{example}[theorem]{Example}
\newtheorem{exercise}[theorem]{Exercise}
\newtheorem{lemma}[theorem]{Lemma}
\newtheorem{notation}[theorem]{Notation}
\newtheorem{problem}[theorem]{Problem}
\newtheorem{proposition}[theorem]{Proposition}
\newtheorem{remark}[theorem]{Remark}
\newtheorem{solution}[theorem]{Solution}
\newtheorem{summary}[theorem]{Summary}
\newenvironment{proof}[1][Proof]{\textbf{#1.} }{\ \rule{0.5em}{0.5em}}
\setlength{\textwidth}{7.0in}
\setlength{\oddsidemargin}{-0.35in}
\setlength{\topmargin}{-0.5in}
\setlength{\textheight}{9.0in}
\setlength{\parindent}{0.3in}
\begin{document}

\begin{flushright}
\textbf{Ziad Arafat \\
FEB 26, 2017}
\end{flushright}

\begin{center}
\textbf{CS 111: CS Principles \\
Homework I} \\
\end{center}

\section{Question 1}

\begin{Large}
\textbf{Algorithm for doing my homework}
\end{Large}


\begin{lstlisting}
Get Coffee or iced tea.
Play ambient music.
Get the assigned homework.
Get the page numbers and question numbers.
Get CS book.
Open CS Book to those pages.
While the homework is not complete:
	While the problem is not complete:
		Try to solve the problem.
		If cup is empty:
			Refill coffee or tea.
		
		Sip coffee or tea.
	Move on to the next problem.
	
Close CS book.
Put CS book in breifcase.
Compile the document and submit.
Pause ambient music.
Recycle coffee/tea container.
End homework session.
\end{lstlisting}

This algorithm meets the criteria in the book because it is largely unambigious to someone who is familiar with the items mentioned and it is effectively 'computable' by a human carrying out the steps. It also produces a result being the complete homework document in a finite amount of time. Thus it meets the books definition of an algorithm\newline

\begin{large}
Checklist
\end{large}

\begin{todolist}
\item unambiguous
\item effectively computable
\item produces a result
\item halts in a finite amount of time
\end{todolist}

\clearpage

\section{Question 5}

I made a table to show the values at each step

\begin{table}[h]
\centering
\caption{Steps and Values}
\label{my-label}
\begin{tabular}{@{}l|l|l|l|l|l@{}}
\toprule
Step & Carry & $C_3$ & $C_2$ & $C_1$ & $C_0$ \\ \midrule
1    & 0     & 0   & 0   & 0   & 0   \\
2    & 0     & 0   & 0   & 0   & 0   \\
3    & 0     & 0   & 0   & 0   & 0   \\
$4_0$  & 0     & 0   & 0   & 0   & 18  \\
$5_0$  & 1     & 0   & 0   & 0   & 8   \\
$6_0$  & 1     & 0   & 0   & 0   & 8   \\
$4_1$  & 1     & 0   & 0   & 7   & 8   \\
$5_1$  & 0     & 0   & 0   & 7   & 8   \\
$6_1$  & 0     & 0   & 0   & 7   & 8   \\
$4_2$  & 0     & 0   & 1   & 7   & 8   \\
$5_2$  & 0     & 0   & 1   & 7   & 8   \\
$6_2$  & 0     & 0   & 1   & 7   & 8   \\
7    & 0     & 0   & 1   & 7   & 8   \\
8    & 0     & 0   & 1   & 7   & 8   \\ \bottomrule
\end{tabular}
\end{table}

\section{Question 8}
It would not be effectively computable with a is equal to 0 because that would make the denominator 0. \newline \newline 
Also any values of a, b, and c where the computation in the square-root equate to -1. Because the square root of -1 is not a real number. We are still technically working with real numbers in that case but the result of the square root will be out of the domain of real numbers. 


\section{Question 11}
It would take $49.19$ billion years to complete. We can find this by calculating the number of permutations of $25$ cities and dividing it by $10,000,000. 25! ∕ 10,000,000$. So it’s definitely not feasible. Without extensive research I cannot find a way to make this at least practical.
\clearpage

\section{Question 13}
It is technically correct in that it will get the correct answer. However it can be simplified. 
\begin{lstlisting}
If x>y and x>z:

	print(x)

	stop 

else if y>z:

	print(y)

	stop

else:

	print(z)
	stop
\end{lstlisting}

\section{Question 18}
dvd char capacity= 5,000,000,000

chars per word = $5$

words per page = $300$

pages per inch = $300$

inches per foot = $12$

shelf space in linear feet = dvd char capacity $\div$ chars per word $\div$ words per page $\div$ pages per inch $\div$ inches per foot

shelf space used in linear feet == 926 ft \newline \newline


\vspace*{0.5in}
\noindent Done in LaTeX
\end{document}